\documentclass[12pt]{book}
\usepackage[utf8]{inputenc}
\usepackage[T1]{fontenc}
\usepackage[czech]{babel}
\usepackage{graphicx}
\usepackage{listings}
\usepackage{mathtools}
\usepackage{amssymb}
\usepackage{geometry}
\usepackage{hyperref}
\usepackage{amsthm}
\geometry{outer=2cm,inner=2cm}
\newtheorem{definice}{Definice}
\newtheorem{veta}{Věta}
\newtheorem{lema}{Lema}
\newtheorem{ozn}{Označení}
\newtheorem{dusledek}{Důsledek}
\newtheorem*{pr}{Příklad}
\newtheorem*{dukaz}{Důkaz}
\newtheorem{pozn}{Poznámka}
\newtheorem{hypoteza}{Hypotéza}
\newtheorem*{umluva}{Úmluva}


\begin{document}
\begin{titlepage}
	\begin{center}
		\textbf{VŠB – Technická univerzita Ostrava}
		
		\vspace{0.5cm}
		
		\textbf{Fakulta elektrotechniky a informatiky}
		
		\vspace{0.5cm}
		
		\textbf{Katedra aplikované matematiky}
		
		\vspace{1.5cm}
		
		Obor: Výpočetní matematika
		
		\vspace{2.5cm}
		
		\textbf{{\huge Nestandardní číselné soustavy}}
		
		\vspace{1cm}
		
		BAKALAŘSKÁ PRÁCE
		
		\vspace{3cm}
		
		\textbf{Vypracoval: Christian Krutsche}
		
		\vspace{0.5cm}
		
		\textbf{Vedoucí: RNDr. Pavel Jahoda, Ph.D.}
		
		\vspace{1cm}
		
		\textbf{2019}
		
		
		
		
	\end{center}
\end{titlepage}

\vspace{15cm}

\textbf{Čestné prohlášení}

\begin{center}
	Prohlašuji, že jsem tuto bakalářskou práci vypracoval samostatně. Veškeré použité podklady, ze
	kterých jsem čerpal informace, jsou uvedeny v seznamu použité literatury a citovány v textu
	podle normy ČSN ISO 690.
\end{center}

\vspace{2.5cm}

V Ostravě dne středa 25.5.2020
Podpis studenta

\newpage

\begin{center}
	\textbf{Poděkování}
	
	Děkuji xx za odborné vedení práce, věcné připomínky, dobré rady a vstřícnost při
	konzultacích a vypracovávání bakalářské práce.
\end{center}

\newpage

\textbf{Abstrakt}

Cílem této práce je prozkoumat různé nestandardní možnosti zápisu či kódování čísel. Kromě všem známých soustav s číselným základem (dvojková, šestnáctková,...), jsou zde i zvláštní soustavy s jiným základem. Práce nám přiblíží spektrum \textbf{//pěti?!!} nestandardních soustav. U každé soustavy se zabývá důkazem jednoznačnosti vyjádření čísel v daném tvaru a důkazem schopnosti vyjádřit libovolně zvolené čísla. Práce zkoumá nejen soustavy s celočíselným základem, ale i se základem iracionálním, či dokonce komplexním.

\newpage

\tableofcontents

\newpage

\chapter{Úvod}
\section{Definice}

\newcommand{\poslbeta}{\{\beta_i\}_{i=1}^{\infty}}
\newcommand{\poslalpha}{\{\alpha_i\}_{i=0}^{\infty}}
\newcommand{\posla}{\{a_i\}_{i=0}^{\infty}}
\newcommand{\poslb}{\{b_i\}_{i=1}^{\infty}}

Připomeňme, že libovolnou podmnožinu $\varphi$ kartézského součinu $A \times B$ nazýváme binární relací(dále jen relací) mezi prvky z množiny $A$ a prvky z množiny $B$. Fakt, že $(a,b)\in \varphi$ budeme značit $\varphi(a) = b$ a $\varphi \subseteq A \times B$ budeme značit $\varphi : A \rightarrow B$, tak jak je to obvyklé u zobrazení, jež jsou speciálními případy relací.\newline\newline
Následující definici jsme převzali z \cite{aa}
\begin{definice}
	Posloupností na množině M rozumíme každou funkci, jejímž definičním oborem je množina $\mathbb{N}$. Posloupnost, která každému $n \in \mathbb{N}$ přiřazuje číslo $a_n \in M$ budeme zapisovat některým z následujících způsobů:
	\begin{itemize}
		\item $a_1, a_2, a_3,\dots$
		\item $(a_n)$
		\item $\{a_n\}_{n=1}^{\infty}$
	\end{itemize}
\end{definice}

\begin{definice}\label{d2} \textbf{(Číselná soustava na tělese)}
	Nechť $(A,+,\cdot)$ je těleso; $\poslalpha$ a $\poslbeta$ jsou posloupnosti na množině $A$; $C\subseteq A$ a $B$ je množina všech posloupností prvků z $C$.
	\textbf{Číselnou soustavou na tělese $(A,+,\cdot)$} o základu $\poslalpha$ a $\poslbeta$ s ciframi z $C$ nazveme libovolnou relaci $\varphi : A \rightarrow B\times B$, kde 	$\varphi(x)=\left(\{a_{i}\}_{i=0}^{\infty},\{b_{i}\}_{i=1}^{\infty}\right)$ právě když
	
	$$x = \sum_{i=0}^{\infty} a_{i}\alpha_{i} + \sum_{i=1}^{\infty} b_{i}\beta_{i}$$
	
	Množinu $C$ označujeme jako \textbf{množinu cifer} číselné soustavy $\varphi$. Budeme používat značení $\varphi(x) = \left(\{a_{i}\}_{i=0}^{\infty},\{b_{i}\}_{i=1}^{\infty}\right) = (\dots a_2,a_1,a_0;b_1, b_2, b_3, \dots)_{\varphi}$ a pokud nebude možno dojít k omylu, pak také $(\dots a_2,a_1,a_0;b_1, b_2, b_3, \dots)_{\varphi} = (\dots a_2,a_1,a_0;b_1, b_2, b_3, \dots) = (\dots a_2a_1a_0,b_1 b_2 b_3 \dots)$
\end{definice}

Všimněme si, že nevyžadujeme, aby $\varphi$ bylo zobrazení. Číselná soustava nemusí vyjadřovat každý prvek z $A$ a ty prvky z $A$, které jsou v relaci $\varphi$, nemusí být vyjádřeny jediným způsobem. Uvažujme například obvyklou desítkovou číselnou soustavu na tělese reálných čísel. Jde o číselnou soustavu, kde $C = \{0, 1, 2, \dots, 9\}$ a základem jsou konstantní posloupnosti na : $\poslalpha = \{10^n\}_{n = 0}^{\infty}$ a $\poslbeta = \{\frac{1}{10^n}\}_{n = 1}^{\infty}$ na množině $C$.

\begin{enumerate}
	
\item[I.] Vyjadřujeme jen nezáporná čísla, např. číslo $x = 1\cdot10^2 + 2\cdot10^1 + 3\cdot10^0 \implies \varphi(x) = (\dots123,000\dots)$, ale $\varphi(-x)$ neexistuje. Pomocí cifer z $C=\{0,\dots,9\}$ při základu $\{10^i\}_{i=0}^{\infty}$ nelze vyjádřit záporné číslo

\item[II.] ($\lfloor x\rfloor$ je celočíselná část reálného čísla $x$)

$$\varphi(1) = \left(  \left\{ \left\lfloor\frac{1}{n+1} \right\rfloor \right\}_{n = 0}^{\infty} , \left\{ 0 \right\}_{n = 0}^{\infty} \right) = \left(\dots 001,000 \dots \right),$$

ale také

$$\varphi(1) = \left(  \left\{ 0 \right\}_{n = 0}^{\infty} , \left\{ 9 \right\}_{n = 0}^{\infty} \right) = \left(\dots 000,999 \dots \right).$$

\end{enumerate}
Analogicky jako na tělese definujeme číselnou soustavu na okruhu.


\begin{definice}\label{d3} \textbf{(Číselná soustava na okruhu)}
	Nechť $(A,+,\cdot)$ je okruh; $\poslalpha$ je posloupnost prvků z $A$; $C\subseteq A$ a $B$ je množina všech posloupností prvků z $C$.
	\textbf{Číselnou soustavou na okruhu $(A,+,\cdot)$} o základu $\poslalpha$ s ciframi z $C$ nazveme libovolnou relaci $\varphi : A \rightarrow B$, kde 	$\varphi(x)= \{a_{i}\}_{i=0}^{\infty}$ právě když
	
	$$x = \sum_{i=0}^{\infty} a_{i}\alpha_{i}$$
	
	Množinu $C$ označujeme jako \textbf{množinu cifer} číselné soustavy $\varphi$. Budeme používat značení $\varphi(x) = \{a_{i}\}_{i=0}^{\infty} = (\dots a_2,a_1,a_0)_{\varphi}$ a pokud nebude možno dojít k omylu, pak také $(\dots a_2,a_1,a_0)_{\varphi} = (\dots a_2,a_1,a_0) = \dots a_2a_1a_0$
\end{definice}

\begin{pozn}
	V Definici \ref{d2} a Definici \ref{d3} předpokládáme, že na tělese, respektive okruhu $(A,+,\cdot)$ jsou definovány nekonečné součty
\end{pozn}

\begin{pozn}
	Všimněme si, že číselná soustava $\varphi$, ať již na tělese, nebo na okruhu, splňuje:
	
	$$\varphi (x_1) = \varphi(x_2) \Rightarrow x_1 = x_2.$$
	
	Proto, je-li $\varphi$ zobrazení, je injektivní.
	Dále můžeme tvrdit, že hodnota $\varphi(x)$ (i v případě, že $\varphi(x)$ není zobrazení) jednoznačně určuje svůj vzor $x$, ale, jak jsme viděli výše, $x$ nemusí jednoznačně určovat svůj obraz $\varphi(x)$.
	
\end{pozn}

\begin{definice}
	Jestliže pro číselnou soustavu $\varphi$ na tělese $(A,+,\cdot)$ platí, že $\varphi$ je zobrazení, pak tuto soustavu nazveme \textbf{jednoznačnou číselnou soustavou na tělese $(A,+,\cdot)$}. Analogicky, jestliže pro číselnou soustavu $\varphi$ na okruhu $(A,+,\cdot)$ platí, že $\varphi$ je zobrazení, pak tuto soustavu nazveme \textbf{jednoznačnou číselnou soustavou na okruhu $(A,+,\cdot)$}.
\end{definice}

Jednoznačnou číselnou soustavou na tělese (respektive okruhu), tedy nazveme každou číselnou soustavu v níž dokážeme vyjádřit libovolný prvek tělesa (okruhu) nejvýše jedním způsobem.

\begin{umluva}\label{u1} \textbf \newline
	\begin{itemize}
		\item Nechť $\poslalpha$ a $\poslbeta$  jsou posloupnosti, které jsou základem číselné soustavy na tělese $(A,+,\cdot)$. Jestliže $\exists n \in \mathbb{A}$, pro které platí:
		      $$ (\forall i \in \mathbb{N} : \alpha_i = n^i,\beta_i = n^{-i}) \land (\alpha_{0} = 1),$$
		      pak prvek \textbf{n} také nazýváme základem této číselné soutavy (1 označuje neutrální prvek tělesa $(A,+,\cdot)$ vzhledem k násobení).
		\item Nechť $\varphi(x) = (\posla,\poslb)$ je číselná soustava na tělese o základu \textbf{n}.\newline Pokud $(\exists n_1 \in \mathbb{N}) (\forall m \in \mathbb{N},m>n_1):a_m = 0$, a pokud $(\exists n_2 \in \mathbb{N}) (\forall m \in \mathbb{N},m>n_2):b_m = 0$, budeme zapisovat: $\varphi(x) = (a_{n_1} \dots a_0,b_1 \dots b_{n_2})_n$
		\item V případě n=10 píšeme pouze $\varphi(x) = a_{n_1} \dots a_0,b_1 \dots b_{n_2}$
		\item Nechť $\poslalpha$ je posloupnost, která je základem číselné soustavy na okruhu $(A,+,\cdot)$ s jedničkou. Jestliže $\exists n \in \mathbb{A}$, pro které platí:
		      $$ (\forall i \in \mathbb{N}_0 : \alpha_i = n^i)$$
		      pak prvek \textbf{n} nazýváme také základem této číselné soutavy $(n^0=1$ je jedničkou v okruhu $(A,+,\cdot))$.
		%\item Neexistuje-li takový prvek, můžeme pro jednoduchost označit základem soustavy definovaný znak charakterizující tuto soustavu. Např. pro faktoriálovou soustavu bude základem znak \textbf{!} a pro Fibonacciho soustavu bude základem znak \textbf{F}
		%TODO vyresit u jednotlivych soustav
		\item Nechť $\varphi(x) = (\posla)$ je číselná soustava na okruhu o základu \textbf{n}.\newline Pokud $(\exists n_1 \in \mathbb{N}) (\forall m \in \mathbb{N},m>n_1):a_m = 0$, budeme zapisovat: $\varphi(x) = (a_{n_1} \dots a_0)_n$
		%\item Jestliže je číselná soustava $\varphi$ na okruhu $\mathbb{N}_0$ jednoznačnou číselnou soustavou na okruhu $\mathbb{N}_0$, pak můžeme $D(\varphi)$ rozšířit na $\mathbb{Z}$ a zapisovat následovně: $\varphi(x) = 	\begin{cases} (a_{n_1} \dots a_0)_n &x\ge0 \\
		%-(a_{n_1} \dots a_0)_n & x<0 \end{cases}$
		%TODO vyresit u jednotlivych soustav
		\item Pokud zmíníme, že číslo z je v relaci s posloupností $a_n$ pro číselnou soustavu na okruhu o základu $\{\alpha_i\}_{i=0}^\infty$, pak dle definice číselné soustavy na okruhu jistě platí $z = \sum_{i=0}^{\infty} a_{i}\alpha_{i}$
	\end{itemize}
\end{umluva}

\newpage
\section{Názorné příklady}
Pro lepší představu definice číselné soustavy si ji předveďme na příkladu
\begin{pr}
	Uvažujme těleso reálných čísel $(\mathbb{R},+,\cdot)$. Tj. zvolili jsme A = $\mathbb{R}$. Obvyklý desetinný zápis reálných čísel je vlastně číselná soustava na tělese $(\mathbb{R},+,\cdot)$ o základu $\poslalpha=\{{10^i\}}_{i=0}^{\infty}$, $\poslbeta=\{{10^{-i}\}}_{i=1}^{\infty}$ a C = $\{0,1,2,3,4,5,6,7,8,9\}$. Podle dohody můžeme říci, že jde o číselnou soustavu na tělese o základu 10 a platí:
	$$\varphi(3\cdot10^2+2\cdot10^1+5\cdot10^0+6\cdot10^{-1})=(\posla,\poslb),$$kde
	$\posla=(5,2,3,0,0,\dots)$ a $\poslb = (6,0,0,\dots)$.\newline Podle Úmluvy \ref{u1} můžeme psát $$\varphi(3\cdot10^2+2\cdot10^1+5\cdot10^0+6\cdot10^-1)=325.6$$
	Označme $x=3\cdot10^2+2\cdot10^1+5\cdot10^0+6\cdot10^{-1}$\newline
	$\varphi(-x) = \varphi(-(3\cdot10^2+2\cdot10^1+5\cdot10^0+6\cdot10^{-1}))$ neexistuje, ale prvek $-x$ ovykle značíme $-x=-\varphi(x)=-325.6$, neboť obvykle nerozlišujeme mezi číslem a jeho ciferným zápisem, tj. mezi $-x$ a $\varphi(-x)$.
	
	
	%\textbf{Desítková soustava}\\
	%Pro tuto (všem dobře známou) soustavu platí:
	%\begin{itemize}
	%	\item[1.] $\poslalpha = \{8^i\}_{i=0}^{\infty}$
	%	\item[2.] $\poslbeta = \{8^{-i}\}_{i=1}^{\infty}$
	%	\item[3.]Podle \textbf{Definice 3} je $\textbf{n} = 8$ základ naší soustavy
	%	\item[4.]$A = \mathbb{R} \Rightarrow (A,+,\cdot) = (\mathbb{R},+,\cdot)$
	%	\item[5.]$C = \{0, 1, 2, ... 7\}$
	%	\item[6.]$B = \{0, 1, 2, ... 7, 10, 11, ... 17, 20, 21, ... 77, 100, ...\}$
	%	\item[7.]$\varphi(113_{10}) = ....000161,000... = 161_8$
	%	\item[8.]$ 113 = \sum_{i=0}^{\infty} a_{i}\alpha_{i} + \sum_{i=1}^{\infty} b_{i}\beta_{i}$ \newline $a_0=1 \land a_1=6 \land a_2=1 \land (\forall i \in \mathbb{N}-\{1,2\}): a_i=0 \land (\forall i \in \mathbb{N}): b_i=0$ \newline
	%	$ \alpha_0 = 1 \land \alpha_1 = 8 \land \alpha_2 = 64$\newline
	%	$ \Rightarrow 113 = 1*1 + 6*8 + 1*64$
	%	\end{itemize}
\end{pr}

%\newpage
%Pozorování:

%Mohutnost C musí být co nejmenší potřebná pro vyjádření celého A.

%Je-li nosná množina tělesa $A \subseteq \mathbb{Z}$, pak posloupnoust $\forall i \in \mathbb{N} : b_i = 0$ (tj. je není třeba vyjadřovat zlomkovou část čísla)\newline \newline
%Je li základem soustavy racionální číslo či posloupnost čísel
%$$\forall i \in \mathbb{N} : \alpha_i \in \mathbb{Q} \land \beta_i \in \mathbb{Q}$$
%pak:
%$$ (\exists k \in \mathbb{N} )(\forall i \in \mathbb{N}, i>k): b_i = 0 $$
%(tj. zlomková část lze vyjádřit konečným počtem prvků \textbf{b})\newline \newline

%Pro každou soustavu platí
%$$ (\exists k \in \mathbb{N} )(\forall i \in \mathbb{N}, i>k): a_i = 0 $$
%(tj. celočíselná část musí být  z konečného počtu prvků \textbf{a})


\chapter{Fibonacciho}

\begin{definice}
	Fibonacciho číselná soustava je číselná soustava na okruhu $(\mathbb{N}_0,+,\cdot)$ o základu $\{F_i\}_{i=0}^{\infty}$(kde $F_i$ je i-tý člen fibonacciho posloupnosti) s množinou cifer $C=\{0,1\}$ a mmnožinou všech posloupností
	$B=\{\{a_n\}_{n=0}^\infty,a_n \in \{0,1\} \}$.\newline
	Fibonacciho číselná soustava je relace:
	$\varphi:\mathbb{N}_0\to B$
\end{definice}
\begin{definice} Fibonacciho posloupnost\newline
	Je nekonečná posloupnost přirozených čísel definována rekurentní formulí:
	$$F_0 = 0,\quad F_1 = 1,\quad F_n = F_{n-1}+ F_{n-2}$$
\end{definice}
\begin{pr}
	$$F_2 = F_1+F_0 =  1 + 0=1$$$$F_3 = F_2+F_1 =  1 + 1= 2$$$$\vdots$$
	$$ \{F_i\}_{i=0}^{\infty} = \{0,1,1,2,3,5,8,13,\dots\}$$
\end{pr}
\begin{umluva}
	V fibonacciho číselné soustavě zapisujeme:
	$\varphi(x) =  (\posla)_F$\newline
	Pokud bychom chtěli vyjádřit záporné číslo (které logicky nemůže patřit do $D(\varphi)$), pak budeme podobně jak jsme zvyklí v desítkové soustavě zapisovat $\varphi(x) = - (\posla)_F$
\end{umluva}
\begin{veta}
	Pro každé $x \in \mathbb{N}_0 \quad \exists\{a_n\}_{n=0}^\infty:x=\sum_{i=0}^{\infty}a_i\cdot F_i\quad$ To jest $D(\varphi)=\mathbb{N}_0$
\end{veta}
\begin{dukaz}\label{fiboDF}
	Protože je fibonacciho posloupnost nekonečná a rostoucí posloupnost, je zřejmé, že pro libovolné $n\in\mathbb{N}_0$ vždy najdeme právě jedno $i\in\mathbb{N}_0$ pro které platí: $F_i\le n<F_{i+1}$
	
	Protože $(\mathbb{Z},+,\cdot)$ je euklidovský obor integrity, jistě existují čísla $z_i\in\mathbb{Z}, a_i\in\{0,1\}$ taková, že:
	$$ z=z_0=z_1\cdot(-2)+a_0,\quad a_0\in\{0,1\}$$
	$$ z_1=z_2\cdot(-2)+a_1,\quad a_1\in\{0,1\}$$
	$$ z_{k-1}=z_k\cdot(-2)+a_{k-1}$$
	$$ z_{n-2}=z_{n-1}\cdot(-2)+a_{n-2}$$
	$$ z_{n-1}=z_n\cdot(-2)+a_{n-1}$$
	$$ z_n=z_{n+1}\cdot(-2)+a_n$$
	\begin{enumerate}
		\item[$\alpha)$] $z_1=0 \implies z=z_0=a_0(-2)^0=a_0\in\{0,1\}$
		\item[$\beta)$]  $z_i \ne 0 \implies \forall k \ge 1:$
		$$|z_k|=\left|\frac{z_{k-1}-a_{k-1}}{-2}\right|\le \frac{|z_{k-1}|+1}{2}$$
		$$|z_{k-1}|\le\frac{|z_{k-2}|+1}{2}\le\frac{\frac{|z_{k-2}|+1}{2}+1}{2}=\frac{|z_{k-2}|}{2^2}+\frac{1}{2}+\frac{1}{4}$$
		$$\vdots$$
		$$|z_k|\le\frac{|z_0|}{2^k}+\sum_{i=1}^{k}\left(\frac{1}{2}\right)^i=\left[\frac{|z_0|}{2^k}+1-\left(\frac{1}{2}\right)^k \right]\to 1 $$
		$$ pro\; k \to \infty $$
		
		Pro dost velké $k_0 \quad |z_{k_0}|\le1,5 \implies \exists k_0:z_{k_0}\in\{1,0,-1\}$
		\begin{enumerate}
			\item[a)] $\underline{z_{k_0}=0}$
			\item[b)]$z_{k_0}=1$
			$$\implies 1=z_{k_0}=z_{k_0+1}\cdot(-2)+a_{k_0}\implies \underline{z_{k_0+1}=0}$$
			\item[c)]$z_{k_0}=-1$
			$$\implies -1=z_{k_0}=z_{k_0+1}\cdot(-2)+a_{k_0}\implies z_{k_0+1}=1\implies \underline{z_{k_0+2}=0}$$
		\end{enumerate}
		$\implies \exists z_{n+1}$ takové, že $z_{n+1}=0$\newline
		$\implies z_n = a_n \implies z_{n-1}=a_n\cdot(-2)^1+a_{n-1}\implies \dots$
		$$\implies z=z_0=a_n\cdot(-2)^n+a_{n-1}\cdot(-2)^{n-1}+\dots+a_0$$
	\end{enumerate}
\end{dukaz}

\begin{pozn} Algoritmus pro hledání reprezentace čísla v negabinární číselné soustavě\newline
	\begin{enumerate}
		\item Nechť $z$ je číslo, které chceme reprezentovat, $z_0 = z$ a $i=0$ je počáteční hodnota algoritmu
		\item Provedeme následující operaci:\newline $z_i/(-2)=z_{i+1}\;zb.\;a_i$, kde $ a_i\in\{0,1\},
		z_i\in\mathbb{Z}$
		\item Opakujeme operaci dokud $z_{i+1}\ne0$, nechť $n$ je počet iterací. ($n$ je jistě konečné, viz. \ref{negaDF})
		\item $\posla$, kde $a_i=0$ pro každé $i>n$, splňuje požadavek $z=
		\sum_{i=0}^{\infty}a_i\cdot(-2)^i$
	\end{enumerate}
	Pozor! zbytek musí vždy patřit do množiny $\{0,1\}$, proto musíme volit $z_{i+1}$ tak, aby platilo: $$z_i=z_{i+1}\cdot(-2)+a_i$$
\end{pozn}
\begin{pr}
	$$z=13$$
	$$13:(-2)=-6\,zb.\,1$$
	$$-6:(-2)=3\,zb.\,0$$
	$$3:(-2)=-1\,zb.\,1$$
	$$-1:(-2)=1\,zb.\,1$$
	$$1:(-2)=0\,zb.\,1$$
	$$\implies a_0=1, a_1=0,a_2=1,a_3=1,a_4=1$$
	$$ 13 = 1\cdot(-2)^0 + 1 \cdot(-2)^2+1\cdot(-2)^3+1\cdot(-2)^4$$
	$$ 13 = 1 + 4-8+16\quad \checkmark$$
\end{pr}
\begin{veta}
	Jestliže $z=\sum_{i=0}^{\infty}a_i(-2)^i$, pak $(\exists n_0 \in \mathbb{N})(\forall n>n_0):a_n=0$
\end{veta}
\begin{dukaz}
	Předpokládejme, že takové číslo $z$ existuje, pak uvažujme tři případy:
	\begin{itemize}
		\item[$a)$]
		Každý sudý člen posloupnosti $a_n$ má hodnotu 1 a existuje $n_0\in\mathbb{N}$, pro který platí že všechny liché členy posloupnosti dále od tohoto $n_0$ mají hodnotu 0.
		Je zřejmé, že suma diverguje a $z=\infty$, a proto $z \notin \mathbb{Z}$
		\item[$b)$]Každý lichý člen posloupnosti $a_n$ má hodnotu 1 a existuje $n_0\in\mathbb{N}$, pro který platí že všechny sudé členy posloupnosti dále od tohoto $n_0$ mají hodnotu 0.
		Je zřejmé, že suma diverguje a $z=-\infty$, a proto $z \notin \mathbb{Z}$
		\item[$c)$] Posloupnost je nekonečná a pro libovolné liché $n_1$ vždy najdeme sudé $n_2$, kde $ n_2>n_1 \land a_{n_2}=1$
		\begin{center}$ n_2 \; $sudé$ \; \implies (-2)^{n_2} = 2^{n_2}$\end{center}
		$$-\left(\sum_{n=0}^{n_2-1}2^n\right)+(-2)^{n_2}\leq \left(\sum_{n=0}^{n_2-1}a_n(-2)^n\right)+(-2)^{n_2}  \leq\left(\sum_{n=0}^{n_2-1}2^n\right)+(-2)^{n_2}$$
		$$-\left(\frac{2^{n_2}-1}{2-1}\right)+2^{n_2}\leq z \leq\left(\frac{2^{n_2}-1}{2-1} \right)+2^{n_2}$$
		$$1\leq z\leq 2\cdot2^{n_2}-1$$
		$$z\ge1$$
		Analogicky pro libovolné sudé $n_1$ vždy najdeme liché $n_2$ větší, kde $ n_2>n_1 \land a_{n_2}=1$
		\begin{center}$ n_2 \; $liché$ \; \implies (-2)^{n_2} = -2^{n_2}$\end{center}
		$$-\left(\sum_{n=0}^{n_2-1}2^n\right)+(-2)^{n_2}\leq \left(\sum_{n=0}^{n_2-1}a_n(-2)^n\right)+(-2)^{n_2}  \leq\left(\sum_{n=0}^{n_2-1}2^n\right)+(-2)^{n_2}$$
		$$-\left(\frac{2^{n_2}-1}{2-1}\right)-2^{n_2}\leq   z \leq\left(\frac{2^{n_2}-1}{2-1} \right)-2^{n_2}$$
		$$-2\cdot2^{n_2}+1\leq z\leq -1$$
		$$z\le-1$$
		
	\end{itemize}
	Je zřejmé, že ani v posledním případě suma nekonverguje, protože vždy najdeme případ, kdy suma je menší než -1 a zároveň případ, kdy suma je větší než 1 $\implies$ \textbf{spor!}
	
\end{dukaz}

\begin{veta}
	Pro každé $z\in\mathbb{Z}\quad\exists!\{a_n\}_{i=0}^\infty:z=\sum_{i=0}^{\infty}a_i(-2)^i$
\end{veta}
\begin{dukaz}
	Dokazujeme sporem, a proto předpokládejme že existuje celé číslo z, které je v relaci s posloupností $\mathbf{a_n}$ a zároveň v relaci s jinou posloupností $\mathbf{b_n}$. Pokud takové číslo existuje, tak $\varphi(z)$ jistě není zobrazení.
	$$\{a_n\}_{n=0}^\infty \ne \{b_n\}_{n=0}^\infty$$ $$z=\sum_{n=0}^{k}a_n(-2)^n = \sum_{n=0}^{k}b_n(-2)^n$$
	definujme posloupnost $C_n$ splňující: $$C_n = a_n - b_n$$
	$$\sum_{n=0}^k C_n(-2)^n = 0 , C_n \in \{-1, 0 ,1\}$$
	Abychom došli ke sporu, předpokládejme, že $\exists n_0 \in \{0,\dots,k\}: C_n \ne 0$
	\begin{itemize}
		\item[$\alpha)$]$C_{n_0} = 1$
		\begin{itemize}
			\item[I.)] $\mathbf{n_0}$ je liché
			$$-\left(\sum_{n=0}^{n_0-1}2^n\right)+(-2)^{n_0}\leq \left(\sum_{n=0}^{n_0-1}C_n(-2)^n\right)+C_{n_0}\cdot(-2)^{n_2}  \leq\left(\sum_{n=0}^{n_0-1}2^n\right)+(-2)^{n_0}$$
			$$1\leq \left(\sum_{n=0}^{n_0-1}C_n(-2)^n\right)+C_{n_0}\cdot(-2)^{n_2}  \leq 2^{n_0 + 1}-1$$
			Takové číslo je jistě kladné $\implies$ \textbf{z} nemůže být v relaci s posloupností $\mathbf{a_n}$ a zároveň v relaci s posloupností $\mathbf{b_n}$
			\item[II.)]  $\mathbf{n_0}$ je sudé
			$$-\left(\sum_{n=0}^{n_0-1}2^n\right)+(-2)^{n_0}\leq \left(\sum_{n=0}^{n_0-1}C_n(-2)^n\right)+C_{n_0}\cdot(-2)^{n_2}  \leq\left(\sum_{n=0}^{n_0-1}2^n\right)+(-2)^{n_0}$$
			$$-2^{n_0+1}+1\leq \left(\sum_{n=0}^{n_0-1}C_n(-2)^n\right)+C_{n_0}\cdot(-2)^{n_2}  \leq -1$$
			Takové číslo je jistě záporné $\implies$ \textbf{z} nemůže být v relaci s posloupností $\mathbf{a_n}$ a zároveň v relaci s posloupností $\mathbf{b_n}$
			
		\end{itemize}
		\item[$\beta)$]$C_{n_0} = -1$\newline
		Důkaz je úplně stejný, ať je $C_{n_0}$ liché nebo sudé, nikdy se suma rovnat 0 jistě nebude.
	\end{itemize}
	
\end{dukaz}

\begin{veta}
	Negabinární číselná soustava je jednoznačná číselná soustava na okruhu celých čísel
\end{veta}

\begin{pr}
	Pro negabinární číselnou soustavu platí:
	$$\varphi(1\cdot-2^6+1\cdot-2^4+1\cdot-2^1)=(\{a_i\})=(64+16+(-2))$$
	$${a_i}=(0,1,0,0,1,0,1,\dots) $$
	$$78 =(1010010)_{-2}$$
\end{pr}


%\chapter{Zlatý řez}


\chapter{Negabinární číselná soustava}

\begin{definice}
	Negabinární číselná soustava je číselná soustava na okruhu $(\mathbb{Z},+,\cdot)$ o základu -2 s množinou cifer $C=\{0,1\}$.
\end{definice}

Negabinární číselná soustava je tedy relace $\varphi$ mezi celými čísly a posloupnostmi jedniček a nul, kde $z \in \mathbb{Z}$ je v relaci s posloupností $\posla$ právě thedy, když 
$$z=\sum_{i=0}^{\infty}a_i\cdot(-2)^i.$$

Ciferným zápisem celého čísla v negabinární číselné soustavě je posloupnost čísel z množiny $C=\{0,1\}$. Fakt, že $\varphi(z) = \posla$, kde $a_i = 0$ pro $i > k$, budeme symbolicky zapisovat $\varphi(z) = (a_k\,\dots\, a_0)_{-2}$. Pokud nebude moci dojít k omylu, tento zápis ještě zjednodušíme na $z = (a_k\,\dots\, a_0)_{-2}$. Například $5 = (101)_{-2}$, neboť $5 = 1\cdot(-2)^2 + 0\cdot(-2)^1 + 1\cdot(-2)^0$.


Prozkoumáme základní vlastnosti této číselné soustavy. Nejprve ukážeme, že v negabinární číselné soustavě dokážeme najít ciferný zápis pro libovolné celé číslo.

\begin{veta}\label{negaDF}
	Pro každé $z \in \mathbb{Z}$ existuje $\posla$, $a_i\in \{0,1\}$ taková, že
     $z=\sum_{i=0}^{\infty}a_i\cdot(-2)^i$. To jest, $D(\varphi)=\mathbb{Z}$.
\end{veta}

\begin{dukaz}
	Protože $(\mathbb{Z},+,\cdot)$ je euklidovský obor integrity, jistě existují čísla $z_i\in\mathbb{Z}, a_i\in\{0,1\}$ taková, že:

\begin{equation}\label{NBe1}
  \begin{array}{rcl}
z = z_0 & = & z_1\cdot(-2) + a_0   \\
            &   &                  \\
    z_1 & = & z_2\cdot(-2) + a_1   \\
            &   &                  \\
    z_2 & = & z_3\cdot(-2) + a_2   \\
\vdots  &   &                      \\
    z_{k-1} & = & z_{k}\cdot(-2) + a_{k-1}  \\
\vdots  &   &                      \\
  \end{array}
\end{equation}

V případě, že $z_1=0$, je jasné, že $z = a_0 = a_0(-2)^0$, kde $a_0\in\{0,1\}$. Dokazované tvrzení tak v tomto případě platí. Co když ale $z_0 \neq 0$?

Dokážeme, že pro dost velké $n$ je $z_{n+1} = 0$.  Z (\ref{NBe1}) je zřejmé, že pro libovolné $k \in \mathbb{N}$ platí

\begin{equation}\label{NBe2}
|z_k|=\left|\frac{z_{k-1}-a_{k-1}}{-2}\right|\leq \frac{|z_{k-1}|+1}{2} = \frac{|z_{k-1}|}{2} + \frac{1}{2}.
\end{equation}

Z (\ref{NBe2}) plyne, že pro $k-1 \in \mathbb{N}$ také platí

\begin{equation}\label{NBe3}
|z_{k-1}|=\left|\frac{z_{k-2}-a_{k-2}}{-2}\right|\leq \frac{|z_{k-2}|}{2} + \frac{1}{2}.
\end{equation}

Aplikací odhadu (\ref{NBe3}) v (\ref{NBe2}) obdržíme

\begin{equation*}\label{NBe4}
|z_k| \leq \frac{|z_{k-1}|}{2} + \frac{1}{2} \leq \frac{|z_{k-2}|}{2^2} + \frac{1}{2} +  \frac{1}{2^2}.
\end{equation*}

Analogicky můžeme pokračovat a zjistíme, že pro libovolné $k \in \mathbb{N}$ platí

\begin{equation}\label{NBe5}
|z_k| \leq \frac{|z_0|}{2^k}+\sum_{i=1}^{k}\left(\frac{1}{2}\right)^i = \frac{|z_0|}{2^k}+1-\left(\frac{1}{2}\right)^k.
\end{equation}

Vzhledem k tomu, že $\frac{|z_0|}{2^k}+1-\left(\frac{1}{2}\right)^k \to 1$ při $k \to \infty$, je zřejmé, že existuje $k_0 \in \mathbb{N}$ takové, že $|z_{k_0}| \leq 1,5$. To ale znamená, že $z_{k_0} \in \{-1, 0, 1\}$. Rozeberme jednotlivé případy.

\begin{enumerate}
			\item[a)] Jestliže $z_{k_0} = 0$, pak také $z_{k_0+1} = 0$. Hledaným $n$ proto může být $n = k_0$.
		  	\item[b)] Jestliže $z_{k_0} = 1$, pak $z_{k_0} = 1 = z_{k_0+1}\cdot(-2)+a_{k_0}$. Protože $a_{k_0} \in \{0,1\}$,
                      musí platit $z_{k_0+1}=0$.  Hledaným $n$ proto může být $n = k_0$.
            \item[c)] Jestliže $z_{k_0} = -1$, pak $z_{k_0} = -1 = z_{k_0+1}\cdot(-2)+a_{k_0}$. Protože $a_{k_0} \in
                      \{0,1\}$, musí platit $z_{k_0+1}=1$. To ale podle předchozího bodu znamená, že $z_{k_0+2}=0$.  Hledaným $n$ proto může být $n = k_0 + 1$.
\end{enumerate}

Dokázali jsme, že existuje $n \in \mathbb{N}$ takové, že $z_{n+1} = 0$. Z toho plyne, že $z_n = a_n$, odtud $z_{n-1} = a_n\cdot(-2)^1 + a_{n-1}$, \dots, $z = z_0 = a_n\cdot(-2)^n + a_{n-1}\cdot(-2)^{n-1} + \cdots + a_0$. Pokud zvolíme $a_i = 0$ pro $i > n$, pak $z=\sum_{i=0}^{\infty}a_i\cdot(-2)^i$.



\end{dukaz}

Na základě úvah provedených ve výše uvedeném důkazu můžeme vyslovit následující tvrzení.

\begin{veta}\label{NBv2}
Pro každé $z \in \mathbb{Z}$ existuje konečná posloupnost $\{a_i\}_{i=0}^k$, $a_i \in \{0, 1\}$ taková, že
$$
z = \sum_{i=0}^{k}a_i(-2)^i
$$
\end{veta}

\begin{dukaz}
  Z důkazu Věty \ref{negaDF} okamžitě plyne, že prokaždé celé číslo existuje posloupnost $\posla$ s konečným počtem nenulových prvků splňující $z=\sum_{i=0}^{\infty}a_i(-2)^i$. 
  
  
  
\end{dukaz}


 Vzniká otázka, zda může existovat posloupnost $\posla$ splňující $z=\sum_{i=0}^{\infty}a_i(-2)^i$ i s nekonečným počtem nenulových prvků? Jinak řečeno, existuje nějaké celé číslo $z$, které je v relaci $\varphi$ s nějakou nekonečnou posloupností? Odpověď je záporná. V negabinární soustavě můžeme každé celé číslo vyjádřit jen pomocí konečného počtu nenulových cifer.

\begin{veta}\label{NBv2.5}
Nechť $z \in \mathbb{Z}$. Jestliže $z=\sum_{i=0}^{\infty}a_i(-2)^i$, pak posloupnost $\posla$ má konečný počet nenulových členů. 
\end{veta}


\begin{dukaz}
Důkaz provedeme sporem. Předpokládejme, že číslo $z=\sum_{i=0}^{\infty}a_i(-2)^i$, kde posloupnost $\posla$ má nekonečně mnoho nenulových členů. Pak může nastat právě jeden ze tří případů:
	\begin{itemize}
		\item[$a)$]
		      Mezi sudými členy posloupnosti $\posla$  (máme na mysli ta $a_n$, kde $n$ je sudé) je nekonečně mnoho
              těch, které mají hodnotu $1$ a existuje jen konečný počet nenulových lichých členů posloupnosti $\posla$.
		      Je zřejmé, že $z = \sum_{i=0}^{\infty}a_i(-2)^i = \infty$. To je spor s tím, že $z \in \mathbb{Z}$.

		\item[$b)$] Mezi lichými členy posloupnosti $\posla$  (máme na mysli ta $a_n$, kde $n$ je liché) je nekonečně mnoho
              těch, které mají hodnotu $1$ a existuje jen konečný počet nenulových sudých členů posloupnosti $\posla$.
		      Je zřejmé, že $z = \sum_{i=0}^{\infty}a_i(-2)^i = -\infty$. To je spor s tím, že  $z \in \mathbb{Z}$.


		\item[$c)$] Jak mezi lichými, tak mezi sudými členy posloupnosti $\posla$ je nekonečně mnoho těch s nenulovou
                    hodnotou.

                    Uvažujme $n_k$ sudé, takové, že $a_{n_k} = 1$.  Označme částečný součet
                    $$S_{n_k} = \sum_{i=0}^{n_k}a_i(-2)^i.$$

                    Můžeme odhadnout hodnotu $S_{n_k}$:

                    \begin{equation}\label{NBe6}
                    \begin{array}{rcl}
                      S_{n_k} & = & (-2)^{n_k} + \sum_{i=0}^{n_k-1}a_i(-2)^i = \\
                       &  & \\
                              & = & 2^{n_k} + \sum_{i=0}^{n_k-1}a_i(-2)^i \geq \\
                       &  & \\
                              & \geq & 2^{n_k} - \sum_{i=0}^{n_k-1}2^i = \\
                       &  & \\
                              & =    & \displaystyle{2^{n_k} - \frac{2^{n_k} - 1}{2-1} = 1}\\
                       &  & \\
                    \end{array}
                    \end{equation}


                    Nyní uvažujme $n_k$ liché, takové, že $a_{n_k} = 1$ a odhadněme hodnotu $S_{n_k}$:

                    \begin{equation}\label{NBe7}
                    \begin{array}{rcl}
                      S_{n_k} & = & (-2)^{n_k} + \sum_{i=0}^{n_k-1}a_i(-2)^i = \\
                       &  & \\
                              & = &  - 2^{n_k} + \sum_{i=0}^{n_k-1}a_i(-2)^i \leq \\
                       &  & \\
                              & \leq & - 2^{n_k} + \sum_{i=0}^{n_k-1}2^i = \\
                       &  & \\
                              & =    &\displaystyle{ - 2^{n_k} + \frac{2^{n_k} - 1}{2-1} = -1}\\
                       &  & \\
                    \end{array}
                    \end{equation}

       Z (\ref{NBe6}) plyne, že pro $n_k$ sudé, kde $a_{n_k} = 1$ (a takových je podle předpokladu nekonečně mnoho), platí $S_{n_k} \geq 1$. Zároveň z (\ref{NBe7}) plyne, že pro $n_k$ liché, kde $a_{n_k} = 1$ (a takových je podle předpokladu také nekonečně mnoho), platí $S_{n_k} \leq -1$. To by však znamenalo, že suma $\sum_{i=0}^{\infty}a_i(-2)^i$ nekonverguje. Opět docházíme ke sporu.

\end{itemize}
	
\end{dukaz}

Nakonec dokážeme, že každé celé číslo je možné v negabinární číselné soustavě vyjádřit jen jedním způsobem, to jest, existuje právě jeden jeho ciferný zápis.

\begin{veta}\label{NBv3}
	Pro každé $z\in\mathbb{Z}$ existuje jediná posloupnost $\{a_i\}_{i=0}^\infty$, $a_i \in\{0,1\}$ taková, že $z = \sum_{i=0}^{\infty}a_i(-2)^i$.
\end{veta}

\begin{dukaz}
Nejprve dokážeme tvrzení věty pro $z = 0$. Věta \ref{negaDF} říká, že existuje posloupnost $\posla$, $a_i \in \{0, 1\}$ taková, že $ 0 = \sum_{i=0}^{\infty}a_i(-2)^i$.

Věta \ref{NBv2.5} navíc tvrdí, že posloupnost $\posla$ nemůže mít nekonečný počet nenulových členů. Musí proto existovat $k \in \mathbb{N}$ takové, že  $a_i = 0$ pro $i > k$. Odtud

\begin{equation}\label{NBv3e1}
  0 = \sum_{i=0}^{k}a_i(-2)^i
\end{equation}

Předpokládejme, že $a_k = 1$. Stejně jako v důkazu Věty \ref{NBv2.5} využijeme odhad součtu (\ref{NBv3e1}). Nejprve pro $k$ liché :

 \begin{equation}\label{NBv3e2}
 \begin{array}{rcl}
0 = \sum_{i=0}^{k}a_i(-2)^i & = & 1\cdot(-2)^{k} + \sum_{i=0}^{k-1}a_i(-2)^i = \\
 &  & \\
 & = &  - 2^{k} + \sum_{i=0}^{k-1}a_i(-2)^i \leq \\
 &  & \\
 & \leq & - 2^{k} + \sum_{i=0}^{k-1}2^i = \\
 &  & \\
 & =    & \displaystyle{- 2^{k} + \frac{2^{k} - 1}{2-1} = -1}\\
 &  & \\
 \end{array}
 \end{equation}

Dostváme sporné tvrzení $0 \leq -1$. Analogicky pro $k$ sudé :

 \begin{equation}\label{NBv3e3}
 \begin{array}{rcl}
0 = \sum_{i=0}^{k}a_i(-2)^i & = & 1\cdot(-2)^{k} + \sum_{i=0}^{k-1}a_i(-2)^i = \\
 &  & \\
 & = &  2^{k} + \sum_{i=0}^{k-1}a_i(-2)^i \geq \\
 &  & \\
 & \geq & 2^{k} - \sum_{i=0}^{k-1}2^i = \\
 &  & \\
 & =    & \displaystyle{ 2^{k} - \frac{2^{k} - 1}{2-1} = 1}\\
 &  & \\
 \end{array}
 \end{equation}

Opět dostáváme sporné tvrzení. Proto $a_k = 0$. Vedoucím koeficientem se tak stává $a_{k-1}$. Opakováním ůvahy zjistíme, že všechny koeficienty $a_i$ musí být rovny nule. Znamená to, že pro $z = 0$ existuje jediná posloupnost $\{a_i\}_{i=0}^\infty$, $a_i \in\{0,1\}$ taková, že $z = \sum_{i=0}^{\infty}a_i(-2)^i$. Jedná se o posloupnost, kde $\forall\,i\in \mathbb{N}\, :\, a_i = 0$.
\newline

	Nyní předpokládejme, že existuje nenulové celé číslo $z \neq 0$, které je v relaci $\varphi$ s posloupností $\posla$ a také s posloupností $\poslb$.

Z Věty \ref{NBv2.5} plyne, že jak poslosloupnost $\posla$, tak posloupnost $\poslb$, má jen konečně mnoho nenulových členů. Označme $k_1 = \max\{i \in \mathbb{N}\,:\, a_i = 1\}$, $k_2 = \max\{i \in \mathbb{N}\,:\, b_i = 1\}$ a $k = \max\{k_1, k_2\}$ (čísla $k_1$ a $k_2$ jistě existují, neboť $z$ je nenulové). Potom jistě platí:

\begin{equation}\label{NBv3e4}
  z=\sum_{i=0}^{k}a_i(-2)^i = \sum_{i=0}^{k}b_i(-2)^i
\end{equation}

Z (\ref{NBv3e4}) plyne, že 

\begin{equation}\label{NBv3e4}
  0 = \sum_{i=0}^{k}(a_i - b_i)(-2)^i 
\end{equation}

Jak jsme ale výše dokázali, musí pro každé $i \in \{0, \dots, k\}$ platit $a_i - b_i = 0$. To ovšem znamená, že $\posla = \poslb$.

\end{dukaz}


\begin{dusledek}
  Negabinární číselná soustava je jednoznačná číselná soustava na okruhu celých čísel.
\end{dusledek}

\begin{dukaz}
Z Věty \ref{NBv3} okamžitě plyne, že relace $\varphi$ je zobrazení.
\end{dukaz}


Výše uvedené poznatky můžeme shrnout. V negabinární číselné soustavě dokážeme vyjádřit libovolné celé číslo pomocí konečného ciferného zápisu a to jediným způsobem.


\begin{pozn} Algoritmus pro hledání reprezentace čísla v negabinární číselné soustavě je založen na konstrukci popsané v důkazu Věty \ref{negaDF}:
\newline

	\begin{enumerate}
		\item Nechť $z$ je číslo, které chceme reprezentovat, $z_0 = z$ a $i=0$ je počáteční hodnota algoritmu.
		\item Pro pro $i>0$ ze vztahu $z_i=z_{i+1}\cdot(-2)+a_i$ určíme čísla $z_{i+1}$ a $a_i$ tak, aby platilo
              $a_i\in\{0,1\}$.
		\item Algoritmus končí pro $i = n$, kde $z_{n+1} = 0$.
		\item Potom $\posla$, kde $a_i=0$ pro každé $i>n$, splňuje požadavek $z=
		\sum_{i=0}^{\infty}a_i\cdot(-2)^i$.
	\end{enumerate}

	
\end{pozn}

\begin{pr} Vyjádříme číslo $z = 13$ v negabinární číselné soustavě.
$$
\begin{array}{rcrclcl}
z_i &=& z_{i+1}&\cdot&(-2) &+& a_i \\ \hline
&&&&&&\\
13 &=& -6&\cdot&(-2) &+& 1 \\
&&&&&&\\
-6 &=& 3&\cdot&(-2) &+& 0 \\
&&&&&&\\
3 &=& -1&\cdot&(-2) &+& 1 \\
&&&&&&\\
-1 &=& 1&\cdot&(-2) &+& 1 \\
&&&&&&\\
1 &=& 0&\cdot&(-2) &+& 1 \\
&&&&&&\\
\end{array}
$$

A opravdu, $1\cdot(-2)^0 + 1 \cdot(-2)^2+1\cdot(-2)^3+1\cdot(-2)^4 = 1 + 4 - 8 + 16 = 13$. Můžeme proto psát $\varphi(13) = (11101)_{-2}$.
\end{pr}

\section{Sčítání celých čísel v negabinární číselné soustavě}

 Sčítání čísel v negabinární číselné soustavě můžeme provádět podle schématu popsaného v následujících příkladech. 
 
 \begin{itemize}
   \item Zvolme $a = 9$ a $b = 19$. Dle výše uvedeného algoritmu nalezneme ciferný záipis těchto celých čísel $9 =  (11001)_{-2}$ a $19 =  (10111)_{-2}$. Cifry zapíšeme pod sebe do tabulky a do řádku pod nimi naznačíme pomocí čárek, kolik má součet $a+b$ příslušných mocnin čísla $-2$:
 
 \begin{equation}\label{NBprsoucT1}
 \begin{tabular}{|c|c|c|c|c|c|c|c|}
   \hline
   % after \\: \hline or \cline{col1-col2} \cline{col3-col4} ...
    &  & 1 & 1 & 0 & 0 & 1 & a \\ \hline
    &  & 1 & 0 & 1 & 1 & 1 & b \\ \hline \hline
    &  &II & I & I & I &II & a+b \\   
   \hline
 \end{tabular}
 \end{equation}
 
 Třetí řádek Tabulky \ref{NBprsoucT1} znázorňuje fakt, že $a + b = (-2)^4 + (-2)^3 + 0(-2)^2 + 0(-2)^1 + 1(-2)^0 + (-2)^4 + 0(-2)^3 + 1(-2)^2 + 1(-2)^1 + 1(-2)^0 = 2(-2)^4 + (-2)^3 + (-2)^2 + (-2)^1 + 2(-2)^0$.
 
 Tento řádek můžeme dále upravovat a to dle následujících pravidel. Protože $(-2)^{k+1} + 2(-2)^k = 0$, můžeme nahradit:
 
  \begin{equation}\label{NBprsoucT2}
 \begin{tabular}{|c|c|c|c|c|c|c|c|}
   \hline
   % after \\: \hline or \cline{col1-col2} \cline{col3-col4} ...
    &  &  & I & II &  &  & a+b \\ \hline
    &  &  & 0 & 0  &  &  & a+b \\
   \hline
 \end{tabular}
 \end{equation}
 
 Protože $ 2(-2)^k = - (-2)^{k+1} $, můžeme nahradit:

  \begin{equation}\label{NBprsoucT3}
 \begin{tabular}{|c|c|c|c|c|c|c|c|}
   \hline
   % after \\: \hline or \cline{col1-col2} \cline{col3-col4} ...
    &  &  &  0  & II &  &  & a+b \\ \hline
    &  &  & -I  & 0  &  &  & a+b \\
   \hline
 \end{tabular}
 \end{equation}
 
  Protože $ -(-2)^k = (-2 + 1) (-2)^{k} = (-2)^{k+1} + (-2)^{k}$, můžeme nahradit:
 
   \begin{equation}\label{NBprsoucT4}
 \begin{tabular}{|c|c|c|c|c|c|c|c|}
   \hline
   % after \\: \hline or \cline{col1-col2} \cline{col3-col4} ...
    &  &  & -I  &  &  &  & a+b \\ \hline
    &  & I&  I  &  &  &  & a+b \\
   \hline
 \end{tabular}
 \end{equation}
 
 Vzhledem k (\ref{NBprsoucT2}), (\ref{NBprsoucT3}) a (\ref{NBprsoucT4}) můžeme v úpravách Tabulky (\ref{NBprsoucT1}) pokračovat následujícím způsobem:
 
  \begin{equation}\label{NBprsoucT5}
 \begin{tabular}{|c|c|c|c|c|c|c|c|}
   \hline
   % after \\: \hline or \cline{col1-col2} \cline{col3-col4} ...
    &  & 1 & 1 & 0 & 0 & 1 & a \\ \hline
    &  & 1 & 0 & 1 & 1 & 1 & b \\ \hline \hline
    &  &II & I & I & I &II & a+b \\
    &-I&   & I & I &   &   & a+b \\
  I & I&   & I & I &   &   & a+b \\
   \hline
 \end{tabular}
 \end{equation}
 
 Z posledního řádku vyčteme, že $a + b = (1101100)_{-2}$. A opravdu, $(-2)^6 + (-2)^5 + (-2)^3 + (-2)^2 = 64 - 32 - 8 + 4 = 28.$ 
 
 \item Sečtěme nyní čísla $a = -3 = (1101)_{-2}$ a $b = -5 = (1111)_{-2}$:
 
   \begin{equation}\label{NBprsoucT5}
 \begin{tabular}{|c|c|c|c|c|c|c|c|}
   \hline
   % after \\: \hline or \cline{col1-col2} \cline{col3-col4} ...
    &  &   & 1 & 1 & 0 & 1 & a \\ \hline
    &  &   & 1 & 1 & 1 & 1 & b \\ \hline \hline
    &  &   &II &II & I &II & a+b \\
    &  &   &I &   &   &   & a+b \\
   \hline
 \end{tabular}
 \end{equation}
 
 Proto $a + b = (1000)_{-2}$. A opravdu, $(-2)^3 = 8$. 
 
  \end{itemize}


\section{Nalezení čísla opačného v negabinární číselné soustavě}

K danému číslu $z$ hledáme číslo opačné. Je zřejmé, že $-z = (-2 + 1)z$. Můžeme proto určit $-z$ jako soucet čísel $z$ a $-2z$. Násobení číslem $-2$ je lehké, stačí připsat nulu na konec ciferného zápisu.  Předvedeme na příkladech:

\begin{itemize}
  \item Nalezneme číslo opačné k číslu $z = 21 = (10101)_{-2}$.
  \begin{equation}\label{NBprsoucT6}
 \begin{tabular}{|c|c|c|c|c|c|c|c|}
   \hline
   % after \\: \hline or \cline{col1-col2} \cline{col3-col4} ...
    &  & 1 & 0 & 1 & 0 & 1 & z \\ \hline
    &1 & 0 & 1 & 0 & 1 & 0 & -2z \\ \hline \hline
    & I& I & I & I & I & I & -z \\
   \hline
 \end{tabular}
 \end{equation}
  
 Proto $-z = (111111)_{-2}$. A opravdu, $-32 + 16 - 8 + 4 - 2 + 1 = -21$.
  
  \item Nalezneme číslo opačné k číslu $z = 7 = (11011)_{-2}$.
  \begin{equation}\label{NBprsoucT6}
 \begin{tabular}{|c|c|c|c|c|c|c|c|}
   \hline
   % after \\: \hline or \cline{col1-col2} \cline{col3-col4} ...
    &  & 1 & 1 & 0 & 1 & 1 & z \\ \hline
    & 1& 1 & 0 & 1 & 1 & 0 & -2z \\ \hline \hline
    & I&II & I & I &II & I & -z \\ \hline
    &  &   & I &   &   & I & -z \\ \hline
 \end{tabular}
 \end{equation}

 Proto $-z = (1001)_{-2}$. A opravdu, $- 8 + 1 = -7$.
  
  
  
\end{itemize}

\section{Násobení v negabinární číselné soustavě}

Násobení v negabinární soustavě je analogické násobení v desítkové soustavě. Násobíme-li číslo $a = (a_n, \dots, a_0)_{-2}$ číslem $(-2)^k = (1, \underbrace{0, \dots, 0}_{k\text{ nul}})_{-2}$, pak stačí přidat na konec ciferného zápisu $k$ nul:

$$(1, \underbrace{0, \dots, 0}_{k\text{ nul}})_{-2}(a_n, \dots, a_0)_{-2} = (a_n, \dots, a_0, \underbrace{0, \dots, 0}_{k\text{ nul}})_{-2}.$$

Proto pro součin čísel $a = (a_n, \dots, a_0)_{-2}$ a $b = (b_k, \dots, b_0)_{-2}$ platí

\begin{equation}\label{NBnase1}
  ab = \sum_{i=0}^{k} a_i\cdot(b_k, \dots, b_0, \underbrace{0, \dots, 0}_{i\text{ nul}} )_{-2}
\end{equation}

Demonstrujeme na příkladech

\begin{itemize}
  \item Vynásobíme čísla $a = 5 = (101)_{-2}$ a $b = 3 = (111)_{-2}$.
  
  $$(101)_{-2} (111)_{-2} = 1\cdot(111)_{-2} + 0 \cdot(1110)_{-2} + 1\cdot (11100)_{-2} = \cdot(111)_{-2} + \cdot (11100)_{-2}$$
  
  Můžeme sečíst schematicky:
  

 \begin{tabular}{|c|c|c|c|c|c|c|c|}
   \hline
   % after \\: \hline or \cline{col1-col2} \cline{col3-col4} ...
    &  &   &   & 1 & 1 & 1 & $x_1$ \\ \hline
    &  & 1 & 1 & 1 & 0 & 0 & $x_3$ \\ \hline \hline
    &  & I & I &II & I & I & $x_1 + x_3$ \\ \hline
    &  & I &   &   & I & I & $x_1 + x_3$  \\ \hline
 \end{tabular}

   
  Proto $ab = (110011)_{-2} = 16 - 2 + 1 = 15$.
  
  \item Vynásobíme čísla $a = -9 = (1011)_{-2}$ a $b = 11 = (11111)_{-2}$.

  $$(1011)_{-2}(11111)_{-2} = (11111)_{-2} + (111110)_{-2} + (11111000)_{-2}$$

  Můžeme sečíst schematicky:

    
 \begin{tabular}{|c|c|c|c|c|c|c|c|c|c|c|c|c|}
   \hline
   % after \\: \hline or \cline{col1-col2} \cline{col3-col4} ...
    &  &  &  &   &   &   & 1 & 1 & 1 & 1 & 1 & $x_1$ \\ \hline
    &  &  &  &   &   & 1 & 1 & 1 & 1 & 1 & 0 & $x_2$  \\ \hline
    &  &  &  & 1 & 1 & 1 & 1 & 1 & 0 & 0 & 0 & $x_4$  \\ \hline \hline
    &  &  &  & I & I  &II&III&III& II& II& I & $x_1 + x_2 + x_4$  \\ \hline
    &  &  &  & I &    &  &II &I  & I &   & I & $x_1 + x_2 + x_4$ \\ \hline
    &  &  &  & I & I  & I&   &I  & I &   & I & $x_1 + x_2 + x_4$ \\ \hline
 \end{tabular}

  Proto $ab = (11101101)_{-2} = -128 + 64 - 32 -8 + 4 + 1 = - 99$.
\end{itemize}





% TODO neni treba dokazat ze + a * funguji i po zobrazeni??

%1.definice
%2. overeni podminek (korektnost) (zkusit overit surjektivitu)
%3. priklad prevodu tam a zpet, scitani, nasobeni
%4. zobecnit prevod, pripadne scitani,nasobeni
%TODO 5. vyhody, nevyhody

%\chapter{Odmocnina ze dvou??}


%\chapter{Kvaterimaginární (2i)}

%\begin{definice}
%	Kvaterimaginární(2i) číselná soustava je číselná soustava (neinjektivní) na tělese $(\mathbb{C},+,\cdot)$ o základu $2i$, kde množina cifer $C = \{0,1,2,3\}$
%\end{definice}

%Podle Úmluvy \ref{u1} v kvaterimaginární(2i) číselné soustavě zapisujeme:
%$$\varphi(x) = (\posla,\poslb)_{2i}$$


%\begin{pr}
%	Jestliže $\varphi$ je kvaterimaginární číselná soustava, pak platí:
%	$$\varphi(1\cdot(2i)^5+1\cdot(2i)^4+3\cdot(2i)^3+2\cdot(2i)^2+3\cdot(2i)^1+1\cdot(2i)^{-1}+3\cdot(2i)^{-2})=$$
%	$$=\varphi(1\cdot(32i)+1\cdot(16)+3\cdot(-8i)+2\cdot(-4)+3\cdot(2i)^1+1\cdot(\frac{-i}{2})+3\cdot(\frac{-1}{4}))=$$
%	$$=(\{a_i\},\{b_i\})=(7.25+13.5i)$$
%	$${a_i}=(0,3,2,3,1,1...),{b_i}=(1,3,...) $$
%78 = 1010010_{-2}
%\end{pr}

%Důkaz neinjektivity
%$$1.03_{2i} = 0.0003_{2i} = \left(\frac{1}{5}\right)_{10}$$


\chapter{Komplexní}

\begin{umluva}
	Pozor! V této kapitole symbolem $i$ značíme imaginární část komplexního čísla\newline
	Protože $a$ již používame pro značení posloupnosti budeme komplexní číslo $z = a +bi$ značit $z=u+vi$, kde u je celočíselná část a v je imaginární část komplexního čísla.
	\end{umluva}

\begin{definice}
	Komplexní číselná soustava je číselná soustava na okruhu $(\mathbb{Z}[i],+,\cdot)$ o základu $\{(1-i)^j\}_{j=0}^\infty$ s množinou cifer $C=\{0,1\}$\newline
	\newline
	Komplexní číselná soustava je relace:
	$\varphi:\mathbb{Z}[i]\to B$
\end{definice}

\begin{veta}
	Pro každé $z \in \mathbb{Z}[i] \quad \exists\{a_n\}_{n=0}^\infty:z=
	\sum_{j=0}^{\infty}a_j\cdot(1-i)^j$
	To jest $D(\varphi)=\mathbb{Z}[i]$
\end{veta}
\begin{dukaz}\label{kompDF}
	...
	$$\frac{u+vi}{1-i}=\frac{u+vi}{1-i}\cdot\frac{1+i}{1+i} = \frac{(u+vi)\cdot(1+i)}{(1+i)\cdot(1-2)}=\frac{(u-v)+(u+v)i}{2}$$
\end{dukaz}
\begin{pozn} Algoritmus pro hledání reprezentace čísla v komplexní číselné soustavě
	\begin{enumerate}
		\item Nechť $z$ je číslo, které chceme reprezentovat, $z_0 = z$ a $j=0$ je počáteční hodnota algoritmu	
		\item Provedeme následující operaci:\newline
		$x = \frac{(u_j-v_j)+(u_j+v_j)i}{2}$\newline
		Jestliže $x \in \mathbb{Z}[i]$, pak $z_{j+1} = x \quad a_{j+1}=0$\newline
		V opačném případě $z_{j+1} = \frac{(u_j-v_j-1)+(u_j+v_j-1)i}{2}\quad a_{j+1}=1$
		\item Opakujeme operaci dokud $z_{j+1}\ne0$, nechť $n$ je počet iterací. ($n$ je jistě konečné, viz. \ref{kompDF})
		\item $\{a_j\}_{j=0}^{\infty}$, kde $a_j=0$ pro každé $j>n$, splňuje požadavek $z=
		\sum_{j=0}^{\infty}a_j\cdot(1-i)^j$
	\end{enumerate}
\end{pozn}


\chapter{Faktoriálová}

\begin{definice}
	Faktoriálová číselná soustava je číselná soustava na okruhu $(\mathbb{N}_0,+,\cdot)$ o základu $\{(i+1)!\}_{i=0}^\infty$ s množinou cifer $C=\mathbb{N}_0$\newline
	\newline
	Faktoriálová číselná soustava je relace:
	$\varphi:\mathbb{N}_0\to B$
\end{definice}

\begin{veta}
	Pro každé $x \in \mathbb{N}_0 \quad \exists\{a_n\}_{n=0}^\infty:x=
	 \sum_{i=0}^{\infty}a_i\cdot(i+1)!$
	To jest $D(\varphi)=\mathbb{N}_0$
\end{veta}
\begin{dukaz}\label{faktDF}
	\begin{enumerate}
		\item[$\alpha)$] $x=0 \implies \forall i \in \mathbb{N}:a_i=0$
		\item[$\beta)$] Je zřejmé, že pro libovolné $x \in \mathbb{N}$ existuje právě jedno $n\in\mathbb{N}:n!\le x<(n+1)!$\newline
		Protože $(\mathbb{N}_0,+,\cdot)$ ??je?? euklidovský obor integrity, jistě existují čísla $x_i\in\mathbb{N}_0, a_i\in\mathbb{N}_0$ taková, že:
		$$ x = x_0 = a_0 \cdot n! + x_1,\quad a_0 \in \mathbb{N}$$
		$$ x_1 = a_1 \cdot(n-1)! + x_2,\quad a_1 \in \mathbb{N}_0$$
		$$ x_2 = a_2 \cdot(n-2)! + x_3,\quad a_2 \in \mathbb{N}_0$$
		$$ x_{k-1} = a_k \cdot (n-k+1)! + x_{k}$$
		$$ x_{n-1} = a_{n-1} \cdot 1! + x_{n}$$
		$$ x_{n} = a_{n} \cdot 0!$$
		
		$\forall k \ge 1:$
		$$x_k=x_{k-1}-a_n\cdot(n-k+1)!$$
		$a_n \in \mathbb{N}_0 \implies$ buď $a_n = x_{k-1}$, pak jistě
		$$x_k \le x_{k-1}-a_n(n-k+1)! \le 0,$$
		protože $(n-k+1)! \le 1$\newline
		Pro dost velké $k_0: x_{k_0} = 0$
		$$\implies x = x_0 = a_n\cdot(0)!+a_{n-1}\cdot(1)!+\dots+a_0$$
	\end{enumerate}
\end{dukaz}
\begin{pozn} Algoritmus pro hledání reprezentace čísla v faktoriálové číselné soustavě
	\begin{enumerate}
		\item Nechť $x$ je číslo, které chceme reprezentovat, $x_0 = x$ a $i=0$ je počáteční hodnota algoritmu
		\item Najdeme nejvyšší n, pro které platí: $n! < x$		
		\item Provedeme následující operaci:\newline $x_i/(n-i)!=a_{n-i-1}\;zb.\;x_{i+1}$, kde $ a_i,x_i\in\mathbb{N}_0$
		\item Opakujeme operaci dokud $x_{i+1}\ne0$, nechť $n$ je počet iterací. ($n$ je jistě konečné, viz. \ref{faktDF})
		\item $\posla$, kde $a_i=0$ pro každé $i>n$, splňuje požadavek $x=
		\sum_{i=0}^{\infty}a_i\cdot(i+1)!$
	\end{enumerate}
\end{pozn}
\begin{veta}
	Jestliže $x=\sum_{i=0}^{\infty}a_i\cdot(i+1)!$, pak $(\exists n_0 \in \mathbb{N})(\forall n>n_0):a_n=0$
\end{veta}
\begin{dukaz}
	Sporem: \newline Nechť existuje taková posloupnost, že pro libovolné $n$ najdeme vždy $n_0$ větší$:a_{n_0}\ne0$\newline
	Pak $x=\sum_{i=0}^\infty a_i\cdot(i+1)! \notin \mathbb{N}_0$, protože suma řady diverguje k $+\infty$
	\end{dukaz}
\begin{definice}
	Omezená množina velikosti i\newline
	$$C_i=\{k,k\in\mathbb{N}_0,k\le i\}=\{0,\dots,i\}$$
	Např. $C_5 = \{0,1,2,3,4,5\}$
\end{definice}
\begin{veta}
	Pro vyjádření libovolného x $\in\mathbb{N}_0$ nám postačí omezená množina $C_i\subseteq\mathbb{N}_0$. Pro každé i $\in \mathbb{N}_0$ pro faktoriálovou číselnou soustavu platí $C=C_i$. To jest, každé číslo x $\in\mathbb{N}_0$ lze vyjádřit následovně:
	$$x=\sum_{i=0}^{\infty}a_i(i+1)!,\quad a_i\in C_i$$
\end{veta}
\begin{dukaz}
...
\end{dukaz}
\begin{veta}
	Jestliže $a_i \in C_{i+1}$, pak je faktoriálová číselná soustava jednoznačná
\end{veta}
\begin{dukaz}
	...
\end{dukaz}
\begin{umluva}
	V faktoriálové číselné soustavě zapisujeme:
	$\varphi(x) =  (\posla)_!$\newline
	Pokud bychom chtěli vyjádřit záporné číslo (které logicky nemůže patřit do $D(\varphi)$), pak budeme podobně jak jsme zvyklí v desítkové soustavě zapisovat $\varphi(x) = - (\posla)_!$
\end{umluva}

\begin{pr}
	$$z=77$$
	$$77:4!=3\,zb.\,5$$
	$$5:3!=0\,zb.\,5$$
	$$5:2!=2\,zb.\,1$$
	$$1:1!=1\,zb.\,0$$
	$$\implies a_0=1, a_1=2,a_2=0,a_3=3$$
	$$ 77 = 1 \cdot 1! + 2 \cdot 2! + 3\cdot 4!$$
	$$ 77 = 1 + 4 + 72 \quad \checkmark$$
\end{pr}

%\chapter{Eulerovo cislo}

\begin{thebibliography}{2}
	\bibitem{aa} \href{https://homel.vsb.cz/~bou10/MA1/4.pdf}{Jiří Bouchala, Matematická analýza ve Vesmíru, strana 3}
\end{thebibliography}

\end{document} 